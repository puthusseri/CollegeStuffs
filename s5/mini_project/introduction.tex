\chapter{Introduction}
\par
The paper coupon market has been with us almost since shopping was first invented. Practically all retail outlets, big or small, have used coupons at least once in their lives. The primary motivation is either to attract more customers by offering them a discount or to enable them to pay in advance for gift vouchers that are subsequently passed on to third parties. The major disadvantage of the paper based coupen was its cost and difficulties in distribution of the coupens. Also the accounting process becomes extremely time consuming. There are issues related to study the market feedback.  A very high proportion of paper coupons are never returned. Paradoxically, a bigger problem is that no firm evidence can be identified to help explain why some actually are returned! All these problems can be solved by digitizing the coupon industry. But a major problem arises due to the arrival of digital coupons. The coupens are getting manipulated by some individuals, which causes a lot of loss to the merchants. So in many situations the loyalty of the customer has to be cross validated by the retailers. 

\par
With the normal encryption mechanism, up to a certain extent this problem can be solved. But if the attacker was highly skilled with tremendous computation power those techniques wont last long. We approach this problem with the disruptive technology, the blockchain. The retailers can generate coupons for a particular product and can share it with the universe using any social media. The coupens are stored in an open source blockchain platform called Ethereum. With the underlying security mechanism of the ethereum blockchain, our system was tamper-proof, as till date nobody was able to break the blockchain network.
