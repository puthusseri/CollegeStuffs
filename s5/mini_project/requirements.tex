\chapter{Requirement Analysis}
\section{Purpose}
\par
Using coupons is a popular and easy way for shoppers to cut the cost of
groceries, personal care items and other household products, but couponing is not
without its rules and regulations. It is important for couponers to understand the
guidelines and stay within the legal boundaries when using coupons. Failure to
do so could result in coupon fraud, which can lead to criminal charges.

\section{Definition of Coupon Fraud}
The Coupon Information Corporation (CIC) has defined coupon fraud as
occurring \textit{"Whenever someone intentionally uses a coupon for a product that
he/she has not purchased or otherwise fails to satisfy the terms and conditions
for redemption, when a retailer submits coupons for products they have not sold
or that were not properly redeemed by a consumer in connection with a retail
purchase; or when coupons are altered/counterfeited."}
\par
The aim of the project is to build a system that can completely eliminate
the coupon fraud and to ensure the end to end security between at all user level
levels. This document explains various features of the management system and
its requirements.
\par
A blockchain-based digital coupon issuing system can solve all these
problems. In the case of #hashCODE, the verification process is made as simple
as giving in a coupon hash as input and obtaining the coupon details. The hash of
each and every coupon generated by the retailer lives in the blockchain along with
the smart contract. It makes sure that coupon cannot be changed as it would
corrupt data in all other blocks of the blockchain as well. In a blockchain,
individual blocks can only be added with the consent of all other parties. This
prevents the generation of fake coupons.
\\
The major objectives behind this work are as follows:
\begin{itemize}
    \item Elimination of coupon fraud
    \item Soft copies ensure that no coupons are lost 
    \item End to End security at different level of users.
\end{itemize}
\section{Overall Description}
\par
Blockchain coupon verification for retail chain can benefit the retail
marketing in large. The manufacture can provide a trusted platform to their
retailers and customers to verify their discount coupon from anywhere in the
globe. This can happen only through technical integration with current
technology. Blockchain for retail chain can be the correct solution for this. All
retail chain comes under some manufacture, the manufacture publishes some
discount coupons to increase their sales, they induce some validity to it, then they
release the coupon to the customers. The customers can redeem the coupons from
any outlets of the manufacture, where the retailer will collect the coupon verify it
and provide the corresponding discount associated.

\begin{itemize}
\item Manufactures\\
 Manufactures are the one who setups the retail chain. The retail chain can
be spread across the globe, various part of the country or can be within a state.
Our particular product is focusing on a large retail market where the retail outlets
are spread across the globe.
\item Retailers \\
 In most of the scenarios the manufacture provides the full authority of
publishing the coupons to the retailers, where it’s a large market.
The retailer is the one who manages a outlet under the manufacture, so he
is responsible for the sales of the product under that locality. So, he can publish
various coupon according the marketing needs, to increase his sales.
\item Customers\\
Customer simply uses the discount coupon published by the retailers, while
having their purchase.
\end{itemize}


\subsection{Product Functions}
The main functions of the proposed system include:
\begin{itemize}
    \item Retail market setup – Adding the retailers.
    \item Adding the product details.
    \item Coupon code generation.
    \item Publishing the coupon.
    \item Verifying the coupon code at any end level.
\end{itemize}

\subsection{Hardware Requirements}
\begin{itemize}
    \item Intel Core i3 or equivalent processor
    \item 4 GB or more RAM
    \item 750 MHZ or more CPU Speed
    \item 500 GB  or more hard disk space
\end{itemize}

\subsection{Software Requirements}
\begin{itemize}
    \item Linux
    \item Ethereum blockchain 
    \item NodeJS
    \item Web3 framework
\end{itemize}

\section{Functional Requirements}
Functional requirements outline the intended behaviour of the system. This behaviour may be denoted as tasks or functions that the specified system is intended to perform. The proposed system consists of the following parts. They are given below:\\
\subsection{Web Interface} \\
A web interface facilitates the interaction of the users with the system. The
manufacture adds the retailer, retailer generates the discount coupon. A QR code
is used corresponding to each coupon, this QR code get scanned and discount get
redeemed by the customers.
\subsection{Ethereum Blockchain}
Ethereum blockchain is very much similar to the bitcoin network.
Ethereum blockchain provides a platform to build decentralized applications
known as Dapps. Similar to the bitcoin network, Ethereum is purely
decentralised. One of the factors that distinguish Ethereum blockchain from
bitcoin is that it is programmable. In addition to the transactions, each block in
the blockchain contains a code snippet called smart contracts. It helps in bringing
together people and organisations from different dimensions of society without
any third party dependency. Ethereum blockchain contains blocks of transactions.
Each block contains a list of transactions and a code snippet called smart contract.
Ethereum uses an algorithm called proof of work algorithm to verify the entire
network. An important data structure that is used by Ethereum is the Merkle tree.
Each and every transaction in Ethereum is represented by a hash value. Merkle
tree is a tree made of transacttion hash values. Inside a block, two transactions
are paired to form a single hash. Then two paired transactions together form
another hash. This process continues until we get a single hash at the root. The
root of a Merkle tree will be an outcome of the entire transactions within that
block. Smart contacts are the crucial components which live inside the blocks of
blockchain in the form of snippets of code. Solidity is the most popularly used
smart contract programming language. The solidity code is very similar to
javascript. Smart contracts are a set of rules and conditions which has to be
followed during transactions. Smart contracts are an integral part as it eliminates
the need for trusted third parties.

\section{Non Functional Requirements}
Non-Functional requirements define the general qualities of the software product. Non-functional requirement is in effect a constraint placed on the system or the development process. They are usually associated with product descriptions such as maintainability, usability, portability, etc. it mainly limits the solutions for the problem. The solution should be good enough to meet the non-functional requirements. 
\section{Performance Requirements}
\begin{itemize}
    \item Accuracy: Accuracy in the functioning and the nature of user-friendliness should be maintained in the system. 
    \item Speed: The system must be capable of offering speed. 
\end{itemize}
\section{Quality Requirements} 
\begin{itemize}
    \item Transparency: The system provides correct data to all participants
    \item  Scalability: The software will meet all of the functional requirements. 
    \item  Maintainability: The system should be maintainable. It should keep backups to atone for system failures and should log its activities periodically.
    \item  Reliability: The acceptable threshold for the downtime should be as long as possible. i.e.mean time between failures should be as large as possible. And if the system is broken, the time required to get the system back up again should be minimum. 
    \item  Consistency: The data should be consistent and precise The system would need a stable internet connection to store and retrieve data from the blockchain database. 
\end{itemize}


