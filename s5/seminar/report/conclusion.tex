\chapter{Conclusion}
\label{chapter:conclusion}
\begin{Conclusion}

The computational complexity of the GAN mainly depends on the complexity of the generator network. By using the style based generator architecture the computational complexity gets reduced upto a large extent. As the techniques like batch normalization, dropouts etc are not used in this architecture, the features of the images that can be generated can exponentially increase with less time. Also FFHQ database of images generated by the StyleGAN can be widely used for both finding solutions to both research and industrial problems. 
Also one of the main challenges of gan  is controlling their output, i.e. changing specific features such pose, face shape and hair style in an image of a face.   A Style-Based Generator Architecture for GANs (StyleGAN), presents a novel model which addresses this challenge. StyleGAN generates the artificial image gradually, starting from a very low resolution and continuing to a high resolution (1024×1024). By modifying the input of each level separately, it controls the visual features that are expressed in that level, from coarse features (pose, face shape) to fine details (hair color), without affecting other levels.
\end{Conclusion}