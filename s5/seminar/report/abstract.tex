\vspace{2in}
\begin{abstract}
    Generative Adversarial Networks (GANs) are a powerful class of neural networks that are used for unsupervised learning. GANs are basically made up of a system of two competing neural network models which compete with each other and are able to analyze, capture and copy the variations within a data set.Although they are computationally complex as all the possible combinations are tried by the generator to produce realistic data. An alternative generator architecture for GAN was proposed, borrowing from style transfer literature, which reduce the complexity exponentially. The new design prompts a consequently learned, unsupervised separation of high level attributes (e.g., posture and identity when trained on human faces) and stochastic variety in the produced pictures (e.g., spots, hair), and it enables intuitive, scale-specific control of the synthesis. The new generator improves the state-of-the-art in terms of traditional distribution quality metrics, leads to demonstrably better interpolation properties. To measure addition quality and unraveling, two new automated strategies that are pertinent to any generator design was introduced. At long last, the result was an  exceptionally differed and excellent dataset of human appearances called FFHQ. The resolution and quality of images produced by 
    generative method using style transfer was of 1024 * 1024, which can include all the human face features.

\end{abstract} 
