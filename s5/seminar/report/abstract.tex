

\begin{titlepage}
\begin{center}
\textbf{\LARGE{Abstract}}\\[0.5cm] 
\end{center}
\paragraph{}
Generative adversarial networks (GANs) is a type of generative model. It is a powerful class of neural networks that are used for unsupervised learning. A GAN consist of two neural networks, a generator network and a discriminator network. The generator generates the data taking random noise as input. The discriminator has to classify whether data generated by the generator is a real or fake data The generator competes against its adversary, the discriminator. As the training for this generative model progresses, the discriminator learns to classify the fake data accurately, while the generator learns to create realistic samples. An equilibrium is reached when the data created by the generator is indistinguishable from real data. GANs are frequently used in image generation and, they produce sharp images too. A downside for GAN is that it does not have a well−defined loss function, which makes training GANs difficult. Although the whole process  was computationally complex as all the possible combinations are tried by the generator to produce realistic data. An alternative generator architecture for GAN was proposed, borrowing from style transfer literature, which reduce the complexity exponentially. The new generator improves the state-of-the-art in terms of traditional distribution quality metrics, leads to demonstrably better interpolation properties. In the end, the result was an  exceptionally differed and excellent dataset of human appearances called FFHQ. The resolution and quality of images produced by generative method using style transfer was of 1024 * 1024, which was exceptionally sufficient include all the human face features. \\\\
Keywords: Generative algorithms, Adversarial training, Style transfer ,Convolutional Neural Network
\end{titlepage}