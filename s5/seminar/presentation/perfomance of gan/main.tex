\documentclass[10pt]{beamer}

\usetheme[progressbar=frametitle]{metropolis}

\usepackage[T1]{fontenc}
\usepackage{newunicodechar}
%\usepackage[utf8]{inputenc}

\usepackage{subcaption}
\usepackage{adjustbox}
\usepackage{booktabs}
\usepackage[scale=2]{ccicons}

% For pseudo codes
\usepackage{algorithm}
\usepackage[noend]{algpseudocode}
\makeatletter
\def\BState{\State\hskip-\ALG@thistlm}
\makeatother
%

\usepackage{multirow}
\usepackage[none]{hyphenat}
\usepackage{textcomp}
\usepackage{gensymb}
\sloppy 
%\usebackgroundtemplate


\usepackage{pgfplots}
\usepgfplotslibrary{dateplot}

\usepackage{xspace}
\newcommand{\themename}{\textbf{\textsc{metropolis}}\xspace}

\setbeamercolor{background canvas}{bg=white!20}

\title{Perfomance of  Generative Adversarial Networks}
\date{\regno}
\author{Vyshak Puthusseri}
\institute{TVE17MCA054 \\
\text{MCA CET }\\
S5 : 51
}
\date{23th Oct 2019}




\newcommand{\nologo}{\setbeamertemplate{logo}{}} 
\newcommand{\congress}{MCA CET}


\setbeamertemplate{footline}
{
  \leavevmode%
  \hbox{%
  \begin{beamercolorbox}[wd=.5\paperwidth,ht=3.0ex,dp=1ex,center]{author in head/foot}%
    \usebeamerfont{author in head/foot}\insertshortauthor
    
  \end{beamercolorbox}%
  \begin{beamercolorbox}[wd=.5\paperwidth,ht=3.0ex,dp=1ex,center]{title in head/foot}%

    \insertframenumber{} / \inserttotalframenumber\hspace*{1ex}
  \end{beamercolorbox}}%
  \vskip0pt%
}

\makeatother




% Document begin %%%%%%%%%%%%%%%%%%%%%%%%%%%%%%%%%%%%%%%%%%%%%%%%%%%%%%%%%%%%%%%%%%%%
\begin{document}

\maketitle
%%%%%%%%%%%%%%%%%%%%%%%%%%%%%%%%%%%%%%%%%%%%%%%%%%%%%%%%%%%%%




%%%%%%%%%%%%%%%%%%%%%%%%%%%%%%%%%%%%%%%%%%%%%%%%%%%%%%%%%%%%%%

%%%%%%%%%%%%%%%%%%%%%%                                          About Style Based Image Quality                   
\begin{frame}[fragile]{Image Quality }
    
        \begin{itemize}
        \item Comparing with CelebA-HQ with FFHQ based on Frechet inception distances (FID) , a great improvement happens
    \end{itemize}

\end{frame}

%%%%%%%%%%%%%%%%%%%%%%%%%%%%%%%%%%%%%%%%%%%%%%%%%%%%%%%%%%%%%%%%%%%%%%

%%%%%%%%%%%%%%%%%%%%%%                              Inception Score (IS)              
\begin{frame}[fragile]{Inception Score (IS)}
IS uses two criteria in measuring the performance of GAN:
    \begin{itemize}
        \item The quality of the generated images and
        \item their diversity.
    \end{itemize}
\end{frame}

%%%%%%%%%%%%%%%%%%%%%%                              Inception Score (IS)              
\begin{frame}[fragile]{Inception Score (IS)}
    \begin{itemize}
        \item Entropy
        \item If the value of a random variable x is highly predictable, it has low entropy.
        \item In GAN, we want the conditional probability P(y|x) to be highly predictable (low entropy).
        \item i.e. given an image, we should know the object type easily.
    \end{itemize}
\end{frame}

%%%%%%%%%%%%%%%%%%%%%%                              Inception Score (IS)              
\begin{frame}[fragile]{Inception Score - Image Quality}
    \begin{itemize}

        \item So we use an INCEPTION NETWORK to classify the generated images and predict P(y|x) — where y is the label and x is the generated data.

      \begin{figure}[ht]
         \hspace*{-1cm}\includegraphics[width=0.8\linewidth]{is.png} \\ \\ \\ \\ \\
    \end{figure}
   \item Misrepresent the performance if it only generates one image per class. p(y) will still be uniform even though the diversity is low
    \end{itemize}
\end{frame}

%%%%%%%%%%%%%%%%%%%%%%                              FID             
\begin{frame}[fragile]{Fréchet Inception Distance (FID)}
    \begin{itemize}

    \item Use the Inception network to extract features from an intermediate layer.
    \item Model the data distribution for these features using a multivariate Gaussian distribution with mean µ and covariance Σ.
    \item FID between the real images x and generated images g:
         \begin{figure}[ht]
             \hspace*{-1cm}\includegraphics[width=1.0\linewidth]{fid.png} \\ \\ \\ \\ \\
        \end{figure}
        where Tr sums up all the diagonal elements.

    \end{itemize}
\end{frame}
%%%%%%%%%%%%%%%%%%%%%%                              FID             
\begin{frame}[fragile]{Fréchet Inception Distance (FID)}
Lower FID values mean better image quality and diversity
\end{frame}

\end{document}
